\phantomsection
\addcontentsline{toc}{section}{Abstract}
%----------------------------------------------------------------------------------------
%	SUMMARY.
%----------------------------------------------------------------------------------------
\section*{\label{sec:abstract}Abstract}
The mission Willzyx I is a hyperbolic orbit around the Earth and then going into the direction of interstellar space. The eccentricity is greater than 1, which means the orbit is not bound to Earth. The mission duration is primary when coming close to Earth, which means the length is about 3 days.\\

This project was intended to confirm or refute the survivability of the proposed mission Willzyx I with certain purposed design features and requirements. In order to do so, the trajectory of the spacecraft and its interaction with the environment is simulated with encounters in the expected regions it passes through. This is done using SPENVIS (SPace ENVironment System). This is a program to model the space environment and its effect, including cosmic rays, natural radiation belts, solar energetic particles, plasmas, gases and micro-particles.\\

First, the intended trajectory of Willzyx I was simulated. After this data was obtained, the regions around Earth that the spacecraft would go through and the environment present in such regions. Then the shielding of the solar arrays was determined, the proposed shielding for the memory device was also studied. Finallt, the single even upsets were then estimated and calculated.\\

After running the necessary simulations using SPENVIS and MATLAB, it can be conclude that the proposed design is well suited for its intended mission with no major changes required.The interactions of the spacecraft with the environment it would encounter was simulated in order to optimize some parameters, like the shielding of the solar arrays which was found to be optimal at 65$\mu m$ thick. The shielding of the memory was found to be more than enough to comply with the mission requirements. \\

In order to have the least amount of change to get single event upsets (SEU's), the CMOS chip has been chosen. This chip uses less energy then a TTL chip and has a higher noise margin. It is expected to have $1.51 \cdot 10^{-2}$\,bits/day, with a shielding of 0.37\,cm of aluminium. This means that the chance can be neglected.
